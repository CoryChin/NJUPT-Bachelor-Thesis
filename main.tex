\documentclass[bachelor]{njupthesis}

%%%%%%%%%%%%%%%%%%%%%%%%%%%%%%%%%%%%%%%%%%%%%%%%%%%
%杂七杂八的补充宏包
\usepackage{xpinyin}
\xpinyinsetup{ratio = .4,font =\ubufont,multiple={\color{black}}}
\usepackage{tcolorbox}
\usepackage{tikz}
\usepackage{color}
\usepackage{xcolor}
% \usepackage{eso-pic}
% \usepackage{mathrsfs}
% \usepackage{setspace}
% \usepackage{array}
% \usepackage{graphicx}   % standard LaTeX graphics tool
%                         % when including figure files
% \usepackage{latexsym}
% \usepackage{zxjatype}
% \usepackage{xeCJK}



%%%%%%%%%%%%%%%%%%%%%%%%%%%%%%%%%%%%%%%%%%%%%%%%%%%
%证毕符号的添加(黑色方块)
\renewcommand{\qed}{\hfill $\blacksquare$}

%%%%%%%%%%%%%%%%%%%%%%%%%%%%%%%%%%%%%%%%%%%%%%%%
\title{xxxxxxxx}
\author{xxx}
\advisor{xxx}
\school{xxx}
\major{xxxxx}
\studentclass{xxx}
\studentid{xxx}
\graduateyear{2023}
\begindate{2023年3月11日}
\finishdate{2023年6月14日}

\begin{document}

% ================函数公式缩写==================
\newcommand{\xon}{x_1,x_2,\cdots,x_n}
\newcommand{\yom}{y_1,y_2,\cdots,y_m}
\newcommand{\bmx}{\bm x}
\newcommand{\bmy}{\bm y}
\newcommand{\bmu}{\bm u}
\newcommand{\bmF}{\bm F}
\newcommand{\bmf}{\bm f}
\newcommand{\bmJ}{\bm J}
\newcommand{\pupxpupy}{\frac{\partial^2 u}{\partial x^2}+\frac{\partial^2 u}{\partial y^2}}
\newcommand{\pupt}{\frac{\partial^2 u}{\partial t^2}}
\newcommand{\pupxon}{\frac{\partial^2 u}{\partial {x_1}^2}+\frac{\partial^2 u}{\partial {x_2}^2}+\cdots+\frac{\partial^2 u}{\partial {x_n}^2}}
\newcommand{\pp}[2]{\frac{\partial #1}{\partial #2}}
\renewcommand{\implies}{\Longrightarrow}%蕴含符号
\renewcommand{\to}{\rightarrow}%右箭头
\renewcommand{\iff}{\Longleftrightarrow}%等价
\newcommand{\Lim}[1]{\lim \limits_{#1}}%极限
\newcommand{\Int}[2]{\int_{#1}^{#2}}%积分
\newcommand{\Sum}[2]{\sum \limits_{#1}^{#2}}%求和
\newcommand{\bcup}[2]{\bigcup \limits_{#1}^{#2}}%大并集
\newcommand{\bcap}[2]{\bigcap \limits_{#1}^{#2}}%大交集
\newcommand{\union}{\cup}%集合的并集
\newcommand{\intersect}{\cap}%集合的交集
\newcommand{\intpi}{\int_{-\pi}^{\pi}}%-pi到pi的积分

\newcommand{\cl}[1]{\overline{#1}} % 闭包


% =============希腊字母缩写===================
\renewcommand{\a}{\alpha}
\renewcommand{\b}{\beta}
\newcommand{\dlt}{\delta}
\newcommand{\Dlt}{\Delta}
\newcommand{\eps}{\varepsilon}
\newcommand{\lam}{\lambda}
\newcommand{\s}{\sigma}
\newcommand{\omg}{\omega}
\newcommand{\Omg}{\Omega}
\newcommand{\Ga}{\Gamma}
\newcommand{\ga}{\gamma}
%\newcommand{\th}{\theta}
\newcommand{\vphi}{\varphi}
\newcommand{\phe}{\varphi}
\newcommand{\cta}{\theta}


% ================cal字体缩写==================
\newcommand{\Cal}[1]{\mathcal{#1}}
\newcommand{\CalA}{\mathcal{A}}
\newcommand{\CalB}{\mathcal{B}}
\newcommand{\CalC}{\mathcal{C}}
\newcommand{\CalD}{\mathcal{D}}
\newcommand{\CalE}{\mathcal{E}}
\newcommand{\CalF}{\mathcal{F}}
\newcommand{\CalG}{\mathcal{G}}
\newcommand{\CalH}{\mathcal{H}}
\newcommand{\CalI}{\mathcal{I}}
\newcommand{\CalJ}{\mathcal{J}}
\newcommand{\CalK}{\mathcal{K}}
\newcommand{\CalL}{\mathcal{L}}
\newcommand{\CalM}{\mathcal{M}}
\newcommand{\CalN}{\mathcal{N}}
\newcommand{\CalO}{\mathcal{O}}
\newcommand{\CalP}{\mathcal{P}}
\newcommand{\CalQ}{\mathcal{Q}}
\newcommand{\CalR}{\mathcal{R}}
\newcommand{\CalS}{\mathcal{S}}
\newcommand{\CalT}{\mathcal{T}}
\newcommand{\CalU}{\mathcal{U}}
\newcommand{\CalV}{\mathcal{V}}
\newcommand{\CalW}{\mathcal{W}}
\newcommand{\CalX}{\mathcal{X}}
\newcommand{\CalY}{\mathcal{Y}}
\newcommand{\CalZ}{\mathcal{Z}}


% ================双线字母缩写================
\newcommand{\F}{\mathbb{F}}
\newcommand{\Q}{\mathbb{Q}}
\newcommand{\R}{\mathbb{R}}
\newcommand{\C}{\mathbb{C}}
\newcommand{\N}{\mathbb{N}}
\newcommand{\Z}{\mathbb{Z}}

\newcommand{\T}{\mathbb{T}}
\newcommand{\bbT}{\mathbb{T}}
\newcommand{\bbD}{\mathbb{D}}
\newcommand{\B}{\mathbb{B}} 
\renewcommand{\H}{\mathbb{H}}
\renewcommand{\P}{\mathbb{P}}


% ===============花体缩写====================
\newcommand{\Scr}[1]{\mathscr{#1}}
\newcommand{\ScrA}{\mathscr{A}}
\newcommand{\ScrB}{\mathscr{B}}
\newcommand{\ScrC}{\mathscr{C}}
\newcommand{\ScrD}{\mathscr{D}}
\newcommand{\ScrE}{\mathscr{E}}
\newcommand{\ScrF}{\mathscr{F}}
\newcommand{\ScrG}{\mathscr{G}}
\newcommand{\ScrH}{\mathscr{H}}
\newcommand{\ScrI}{\mathscr{I}}
\newcommand{\ScrJ}{\mathscr{J}}
\newcommand{\ScrK}{\mathscr{K}}
\newcommand{\ScrL}{\mathscr{L}}
\newcommand{\ScrM}{\mathscr{M}}
\newcommand{\ScrN}{\mathscr{N}}
\newcommand{\ScrO}{\mathscr{O}}
\newcommand{\ScrP}{\mathscr{P}}
\newcommand{\ScrQ}{\mathscr{Q}}
\newcommand{\ScrR}{\mathscr{R}}
\newcommand{\ScrS}{\mathscr{S}}
\newcommand{\ScrT}{\mathscr{T}}
\newcommand{\ScrU}{\mathscr{U}}
\newcommand{\ScrV}{\mathscr{V}}
\newcommand{\ScrW}{\mathscr{W}}
\newcommand{\ScrX}{\mathscr{X}}
\newcommand{\ScrY}{\mathscr{Y}}
\newcommand{\ScrZ}{\mathscr{Z}}


% ============= 文字上色 ==============
\newcommand{\red}[1]{\textcolor{red}{#1}}
\newcommand{\cyan}[1]{\textcolor{cyan}{#1}}
\newcommand{\NvBlue}[1]{\textcolor{NavyBlue}{#1}}
\newcommand{\OrangeRed}[1]{\textcolor{OrangeRed}{#1}}
\newcommand{\RedOrange}[1]{\textcolor{RedOrange}{#1}}

% ============== math terminology =============
\newcommand{\Salg}{$\sigma$-algebra} % sigma- algebra


% ==============
\renewcommand{\det}{\mathrm{det}}
\renewcommand{\d}{\mathrm{d}}

%公式缩写新定义命令

\makecover

\begin{chineseabstract}
这是论文摘要。 

需要的话可以用多个段落。

……

\chinesekeyword{}
\end{chineseabstract}

\begin{englishabstract}
With the going on of English abstract......

......

\englishkeyword{}
\end{englishabstract}

\thesistableofcontents%生成目录

\thesischapterexordium


%%%%%%%%%%%%% 正文 %%%%%%%%%%%%%%%%%%
\chapter{绪论}
这是正文了

\section{研究工作的背景与意义}




\section{对xxxxxxx的研究历史与现状}




\section{本文的主要贡献与工作}

本论文主要贡献与工作如下:




\section{本论文的结构安排}

本文的章节结构安排如下:


%第一章:绪论
\input{Mainbody/Conclusion}%第六章:全文总结与展望


%%%%%%%%%%%%%%%%%%%%%%%% 致谢 %%%%%%%%%%%%%%%%%%%%%%%%%%%%%
\thesisacknowledgement

谢谢谢谢谢谢

\vspace{\baselineskip}
\begin{flushright}\noindent
Nanjing, China \hfill {\it 秦辰彬 Chin Chenpin}\\
\hfill {\it 2023.6}
\end{flushright}



%%%%%%%%%%%%%%%%%%%%%% 参考文献 %%%%%%%%%%%%%%%%%%%%%%%%%%%
\thesisloadbibliography[nocite]{reference}

%
% Uncomment following codes to load bibliography database with native
% \bibliography command.
%
% \nocite{*}
% \bibliographystyle{njupthesis}
% \bibliography{reference}
%

\thesisappendix
本论文的写作\LaTeX 模板与答辩演示的幻灯片\LaTeX 模板已上传本人的GitHub仓库,详情请见\underline{\url{https://github.com/CoryChin/NJUPT-Bachelor-Thesis}}. 

排版本论文的\LaTeX 模板以Overleaf-Template中的NJPUPTthesis模板为原型,由本人按照南京邮电大学2023年发布的最新本科生毕业论文的Word文件格式要求进行修改所得。答辩演示的幻灯片\LaTeX 模板修改自清华大学学术幻灯片模板,特此感谢南京邮电大学2018级理学院张欣宇学长为此模板的修改所做的主要工作。

修改后的文档写作\TeX 模板版权系南京邮电大学理学院秦辰彬所有,但因项目地址已开源,故任何有需要学生可随意使用,如有使用请自觉在论文中声明出处并致谢。

\end{document}
